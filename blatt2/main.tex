\documentclass{article}
\usepackage{graphicx} % Required for inserting images
\usepackage{amsmath} % for displaying matrices
\usepackage{multicol} % for displaying things side by side
\usepackage{hyperref} % für hyperlinks

\title{Wissenschaftliches Rechnen1 Blatt2}
\author{Janik Teune}
\date{June 2023}

\newcommand{\dblunderline}[1]{\underline{\underline{#1}}}

\begin{document}

\maketitle

\section{Aufgabe 1}

\subsection{Gauß-Elimination mit Pivoting}

\[
    \left(
    \begin{array}{cc|c}
        0.00025 & 2.32 & 1.387 \\
        10.126 & 1.257 & 0.586 \\
    \end{array}
    \right)
\]

Werte Runden:
\[
    \left(
    \begin{array}{cc|c}
        2.5*10^{-4} & 2.3*10^0 & 1.4*10^0 \\
        1.0*10^1 & 1.3*10^0 & 5.9*10^{-1} \\
    \end{array}
    \right)
\]

Erste und zweite Zeile tauschen:
\[
    \left(
    \begin{array}{cc|c}
        1.0*10^1 & 1.3*10^0 & 5.9*10^{-1} \\
        2.5*10^{-4} & 2.3*10^0 & 1.4*10^0 \\
    \end{array}
    \right)
\]

Ziehe ein Vielfaches der zweiten Zeile von der ersten Zeile ab:
\[
    \frac{1.0*10^1}{2.5*10^{-4}} = 4.0*10^4
\]

\[
    \left(
    \begin{array}{cc|c}
        1.0*10^1 & 1.3*10^0 & 5.9*10^{-1} \\
        0 & -9.2*10^4 & -5.6*10^4 \\
    \end{array}
    \right)
\]

Berechne \(x_2\):
\[
    x_2 = \frac{-5.6*10^4}{-9.2*10^4} = \dblunderline{6.1*10^{-1}}
\]

Berechne \(x_1\):
\[
    x_1 = \frac{5.9*10^{-1} - 1.3*10^0*x_2}{1.0*10^1} =  \dblunderline{-2.0*10^{-2}}
\]

\textbf{Jetzt ohne Pivoting}

\[
    \left(
    \begin{array}{cc|c}
        0.00025 & 2.32 & 1.387 \\
        10.126 & 1.257 & 0.586 \\
    \end{array}
    \right)
\]

Werte Runden:
\[
    \left(
    \begin{array}{cc|c}
        2.5*10^{-4} & 2.3*10^0 & 1.4*10^0 \\
        1.0*10^1 & 1.3*10^0 & 5.9*10^{-1} \\
    \end{array}
    \right)
\]

Ziehe ein Vielfaches der zweiten Zeile von der ersten Zeile ab:
\[
    \frac{2.5*10^{-4}}{1.0*10^1} = -2.5*10^{-5}
\]

\[
    \left(
    \begin{array}{cc|c}
        2.5*10^{-4} & 2.3*10^0 & 1.4*10^0 \\
        0 & 2.3*10^0 & 1.4*10^0 \\
    \end{array}
    \right)
\]

Berechne \(x_2\):
\[
    x_2 = \frac{1.4*10^0}{2.3*10^0} = \dblunderline{6.1*10^{-1}}
\]

Berechne \(x_1\):
\[
    x_1 = \frac{1.4*10^0 - 2.3*10^0*x_2}{2.5*10^{-4}} =  \dblunderline{-1.2*10^1}
\]


\textbf{Die richtig Lösung wäre:}
\begin{multicols}{2}
    \[\hat{x_1} = \frac{1535756}{93968023}\]
    \break
    \[\hat{x_2} = \frac{56178462}{93968023}\]
\end{multicols}

Error with pivoting:

\[
    e_{p1} = \frac{\left(\hat{x_1} - x_1\right)}{\hat{x_1}} \approx 2
\]

\[
    e_{p2} = \frac{\left(\hat{x_2} - x_2\right)}{\hat{x_2}} \approx 0.02
\]


Error without pivoting:

\[
    e_1 = \frac{\left(\hat{x_1} - x_1\right)}{\hat{x_1}} \approx 735
\]

\[
    e_2 = \frac{\left(\hat{x_2} - x_2\right)}{\hat{x_2}} \approx 0.02
\]

Wie man sieht wird der Fehler für \(x_1\) in beiden Fällen ziemlich groß.
Das liegt daran, dass 2 Mantissenbits sehr wenig sind und der Rundungsfehler dadurch sehr groß wird.

Ohne das Pivoting ist der Fehler allerdings nochmal deutlich größer.
Wenn wir zur Berechnung von \(x_1\) schauen sehen wir auch warum das so ist.

Ohne das Pivoting haben wir im Nenner einen Exponenten von \(-4\) und im Zähler einen von \(0\).
Der Unterschied in den Exponenten ist also relativ groß und dadurch wird auch der Rundungsfehler größer.

Mit Pivoting sind die Exponenten die in der Gleichung vorkommen \(-1\), \(0\) und \(1\). 
Damit ist der Unterschied der Exponenten und dann auch der Rundungsfehler kleiner.

Bei \(x_2\) gibt es keinen Unterschied im Fehler.

\subsection{Vergleich Gauß und Cholesky}

Das Cholesky Verfahren und die Gauß-Elimination haben beide Vorteile.

Zuallererst kann man die Gauß-Elimination immer verwenden, wenn das GLeichungssystem linear ist.
Die ist eine relativ schwache Einschränkung.
Das Cholesky-Verfahren hingegen funktioniert nur, wenn das Gleichungssystem zusätzlich noch positiv-semidefinit und symmetrisch ist.

Das heißt allerdings auch, dass man mit dem Cholesky-Verfahren gleichzeitig auch überprüfen kann, ob eine Matrix positiv-semidefinit ist. Wenn man also nur derartige Matrizen erwartet. Dann kann es von Vorteil sein, wenn das Verfahren bei einer Matrix fehlschlägt, die die geforderten Eigenschaften nicht erfüllt.

Sollte die Matrix, die Forderungen der Cholesky Zerlegung erfüllen, dann ist die Zerlegung auch etwas effizienter als das Gauß verfahren.

Quelle: \url{http://www.tm-mathe.de/Themen/html/gaussvscholesky.html}


\section{Ausgleichsrechnung}

\subsection{Normalengleichung}

Eine Matrix der Form \(AA^T\) ist immer positiv-semidefinit und symmetrisch. Damit ist eine derartige Matrix auch für die Cholesky Zerlegung geeignet.

\textbf{Beweis: }
Eine Matrix \(M\) ist positiv-semidefinit, g.d.w. gilt \(x^TMx \geq 0\).
Dies gilt für einen beliebigen Vektor \(x\), wenn \(M=AA^T\) ist, denn:
\[x^TAA^Tx = (A^Tx)^TA^Tx = ||A^Tx||^2 \geq 0\]

Eine Matrix \(M\) ist symmetrisch, g.d.w \(M=M^T\).
Dies gilt ebenfalls für \(M=AA^T\), da:
\[AA^T = (A^T)^TA^T = (AA^T)^T\]

\subsection{Eindeutigkeit der Lösung linearer Gleichungssysteme}

nicht bearbeitet..

\subsection{Abstand von Geraden im Raum}

\[
    g_1 = 
    \begin{pmatrix}
        1 \\ 0.5 \\ 2
    \end{pmatrix}
    + t * 
    \begin{pmatrix}
        0.4 \\ -0.2 \\ 1.1
    \end{pmatrix}
\]

\[
    g_2 = 
    \begin{pmatrix}
        -0.2 \\ 0.7 \\ 1.3
    \end{pmatrix}
    + s *
    \begin{pmatrix}
        -0.35 \\ 1.5 \\ -0.7
    \end{pmatrix}
\]

Zuerst finden wir mit dem Kreuzprodukt den Vektor, der zu den beiden Richtungsvektoren der Geraden orthogonal steht.

\[
    \begin{pmatrix}
        0.4 \\ -0.2 \\ 1.1
    \end{pmatrix}
    \times
    \begin{pmatrix}
        -0.35 \\ 1.5 \\ -0.7
    \end{pmatrix}
    =
    \begin{pmatrix}
        -1.51 \\ -0.105 \\ 0.53
    \end{pmatrix}
\]

Um nun die Punkte auf den Geraden zu bestimmen, die den kürzesten Abstand zueinander haben, spannen wir eine Ebene mit \(g_1\) und dem eben berechneten Vektor auf.

\[
    E = 
    \begin{pmatrix}
        1 \\ 0.5 \\ 2
    \end{pmatrix}
    + t * 
    \begin{pmatrix}
        0.4 \\ -0.2 \\ 1.1
    \end{pmatrix}
    + u * 
    \begin{pmatrix}
        -1.51 \\ -0.105 \\ 0.53
    \end{pmatrix}
\]

Jetzt wollen wir den Punkt bestimmen, wo \(g_2\) auf die Ebene trifft.

\[
    \begin{pmatrix}
        -0.2 \\ 0.7 \\ 1.3
    \end{pmatrix}
    + s *
    \begin{pmatrix}
        -0.35 \\ 1.5 \\ -0.7
    \end{pmatrix}
    =
    \begin{pmatrix}
        1 \\ 0.5 \\ 2
    \end{pmatrix}
    + t * 
    \begin{pmatrix}
        0.4 \\ -0.2 \\ 1.1
    \end{pmatrix}
    + u * 
    \begin{pmatrix}
        -1.51 \\ -0.105 \\ 0.53
    \end{pmatrix}
\]

Durch umstellen erhalten wir ein lineares Gleichungssystem:

\[
    \left(
    \begin{array}{ccc|c}
        0.4 & -1.51 & 0.35 & -1.2 \\
        -0.2 & -1.05 & -1.5 & 0.2 \\
        1.1 & 0.53 & 0.7 & -0.7 \\
    \end{array}
    \right)
\]

Dieses Gleichungssystem können wir wie gewohnt lösen und erhalten:

\begin{multicols}{3}
    \[
        t = -\frac{1333}{2137}
    \]
    \break
    \[
        s = \frac{4512}{10685}
    \]
    \break
    \[
        u = \frac{1136}{2137}
    \]
\end{multicols}

Jetzt können wir mit den Gleichungen der Geraden die gesuchten Punkte berechnen.

\[
    P_1 = 
    \begin{pmatrix}
        1 \\ 0.5 \\ 2
    \end{pmatrix}
    - \frac{1333}{2137} * 
    \begin{pmatrix}
        0.4 \\ -0.2 \\ 1.1
    \end{pmatrix}
    \approx
    \begin{pmatrix}
        0.750491 \\ 0.624754 \\ 1.31385
    \end{pmatrix}
\]

\[
    P_2 = 
    \begin{pmatrix}
        -0.2 \\ 0.7 \\ 1.3
    \end{pmatrix}
    + \frac{4512}{10685} *
    \begin{pmatrix}
        -0.35 \\ 1.5 \\ -0.7
    \end{pmatrix}
    \approx 
    \begin{pmatrix}
        -0.052204 \\ 0.0665887 \\ 1.59559
    \end{pmatrix}
\]

Jetzt berechnen wir die Distanz dieser beiden Punkte:
\[
    ||P_1-P_2|| = 1.01747
\]

\end{document}
